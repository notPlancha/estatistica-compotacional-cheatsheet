\documentclass[a4paper]{article}

% region packages
\usepackage{multicol}
\usepackage{calc}
\usepackage{ifthen}
\usepackage[landscape]{geometry}
\usepackage{hyperref}
\usepackage{blindtext}
\usepackage{xfrac}
\usepackage{amsmath}
\usepackage[portuguese]{babel}
\usepackage[raggedrightboxes]{ragged2e}
\usepackage{booktabs}
\usepackage{array}
\usepackage{tabularray}
\usepackage[none]{hyphenat}
\usepackage{titlesec}
% endregion
\geometry{top=1cm,left=1cm,right=1cm,bottom=1cm} % page margins
\pagestyle{empty} % Turn off header and footer
\setcounter{secnumdepth}{0} % Don't print section numbers
% region Redefine section commands to use less space
\titlespacing{\section}{0pt}{0pt}{0pt}
\titlespacing{\subsection}{0pt}{0pt}{0pt}
\titlespacing{\subsubsection}{0pt}{0pt}{0pt}
% endregion

\setlength{\parindent}{0pt}
\setlength{\parskip}{0pt plus 0.5ex}


% -----------------------------------------------------------------------

\begin{document}
\raggedright
\begin{multicols}{3}
% These lengths are set only within the two main columns
%\setlength{\columnseprule}{0.25pt}
\setlength{\premulticols}{1pt}
\setlength{\postmulticols}{1pt}
\setlength{\multicolsep}{1pt}
\setlength{\columnsep}{2pt}

\begin{center}
  \Large{\textbf{Estatística Computacional}} \\
  \small{Plancha; 105289; CDA2} \\
  \small{Versão 0.1}
\end{center}

\section{Teoria de Probabilidades}

\begin{tblr}{X c X[2]}\SetRow{m}
  Esperiência aleatória & & Processo de observação de fenómenos aleatórios \\
  Fenómenos aleatórios & & Acontecimentos não determináveis \textit{à priori} \\ \SetRow{m} 
  Espaço de resultados & $\Omega$ & Conjunto de todos os resultados possíveis \\ \SetRow{m}
  Acontecimentos & $A, B, C$ & Conjunto de possíveis resultados da experiência aleatória \\ \SetRow{m}
  Resultado da experiência aleatória & $\omega$ & $A$ realizou-se se $\omega \in A$ 
\end{tblr}

\subsection{Álgebra dos acontecimentos}
\subsubsection{União}
$$A \cup B = \{ \omega : \omega \in A \lor \omega \in B \}$$
\subsubsection{Intersecção}
$$A \cap B = \{ \omega: \omega \in A \land \omega \in B \}$$
\subsubsection{Diferença}
$$A - B = A \setminus B = \{ \omega: \omega \in A \land \omega \notin B \}$$
$$\Omega - B = \overline{B} = \{ \omega: \omega \in \Omega \land \omega \notin B \}$$
\subsubsection{Propriedades}
\begin{tblr}{l r}
  Comutativa & {$A \cup B = B \cup A$ \\ $A \cap B = B \cap A$} \\
  Associativa & {$A \cup (B \cup C) = (A \cup B) \cup C$ \\ $A \cap (B \cap C) = (A \cap B) \cap C$} \\
  Distributiva & {$A \cup (B \cap C) = (A \cup B) \cap (A \cup C)$ \\ $A \cap (B \cup C) = (A \cap B) \cup (A \cap C)$} \\
  Idempotência & {$A \cup A = A$ \\ $A \cap A = A$} \\
  Lei do Complemento & {$A \cup \overline{A} = \Omega$ \\ $A \cap \overline{A} = \emptyset$}
\end{tblr} %column break
\subsubsection{Probabilidades (Cont)}
\begin{tblr}{l r}

  Elemento Neutro & {$A \cup \emptyset = A$ \\ $A \cap \Omega = A$} \\
  Elemento Absorvente & {$A \cup A = A$ \\ $A \cap \emptyset = \emptyset$} \\
  Leis de Morgan & {$\overline{A \cup B} = \overline{A} \cap \overline{B}$ \\ $\overline{A \cap B} = \overline{A} \cup \overline{B}$}
\end{tblr}

\subsection{Probabilidades}
TODO





















\section{temp}
\blinddocument
\end{multicols}
\end{document}
